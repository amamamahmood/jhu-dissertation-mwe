\chap{Abstract}
%
% REPLACE THIS WITH ABSTRACT TEXT
The unique social presence of robots can be leveraged in learning situations to increase comfortability and engagement of kids, while still providing instructional guidance. When and how to interfere to provide feedback on their mistakes is still not fully clear. One effective feedback strategy used by human tutors is to implicitly inform the students of their errors rather than explicitly providing corrective feedback. This essay explores if and how a social robot can be utilized to provide implicit feedback to a user who is performing spatial visualization tasks. We explore impact of implicit and explicit feedback strategies on user's learning gains, self-regulation and perception of robot during 3D block building tasks in one-on-one child-robot tutoring.\\
We demonstrate a realtime system that tracks the assembly of a 3D block structure using a RealSense\textsuperscript\textregistered{} RGB-D camera. The system allows three control actions: \emph{Add, Remove and Adjust} on $2 \times 4 $ and $2 \times 4 $ blocks (similar to Lego\textsuperscript\textregistered{}) of four basic colors (red, blue, green and yellow) to manipulate the 3D structure in the play area. 3D structures can be authored in the \emph{Learning mode} for system to record, and tracking enables the robot to provide selected feedback in the \emph{Teaching mode} depending on the type of mistake made by the user. Proposed perception system is capable of detecting five types of mistakes i.e., mistake in: shape, color, orientation, level from base and position of the block.\\
The feedback provided by the robot is based on mistake made by the user. Either implicit or explicit feedback, chosen randomly, is narrated by the robot. Various feedback statements are designed to implicitly inform the user of the mistake made. Two robot behaviours have been designed to support the effective delivery of feedback statements. Nodding is used in conjunction with continuers i.e., \emph{Go on, Continue, Good, Hmm and Right} when the user action results in a correct outcome. Whereas, referential gaze is employed along with the requisite feedback statement whenever the user commits a mistake during the assembly task. \\
We conducted an exploratory study to evaluate our robot assisted 3D block building system to augment spatial visualization skills with one participant. We found that the system was easy to use. The robot was perceived as trustworthy, fun and interesting. Intentions of the robot are communicated through feedback statements and its behaviour i.e., nodding and referential gaze. Our goal is to explore that the suggestion of mistakes in implicit ways can help the users self-regulate and scaffold their learning processes. However, we do not have enough evidence to support this from our exploratory study. \\
Furthermore, we discuss shortcomings of our system, compare it to few existing systems, discuss design implications and ways to improve it as future work. The observations from the study with one elementary-aged study are contribution towards our future endeavours in the field of social robotics and education. 
%
% Add Thesis Readers
\section*{Thesis Readers}
\begin{singlespace}
%
\noindent Dr.~Chien-Ming Huang (Primary Advisor)\\
\indent \indent Assistant Professor\\
\indent \indent Department of Computer Science\\
\indent \indent Johns Hopkins University\\


%
%\noindent First Lastname \\
%\indent \indent Associate Professor\\
%\indent \indent Affiliation1, and\\
%\indent \indent Department1 \& Department2 at\\
%\indent \indent Johns Hopkins Bloomberg School of Public Health \\
%
\end{singlespace}