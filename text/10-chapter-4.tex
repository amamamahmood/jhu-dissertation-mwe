\chapter{Exploratory Study}
\label{chap:four}
%\newpage

% Introduction
We conducted a study to explore how various feedback strategies (implicit and explicit) can impact user's ability to self-regulate and scaffold learning process. We want to understand if and how a robot tutor can be of assistance during a spatial visualization task and how specific robot behaviour will impact user's perception of the robot. Various aspects of this exploratory study are explained below:
\sect{Participant}
Our participant was an elementary-aged female student. She is currently in fourth grade. Her past experience with technology in the classroom is limited to using tablet computers. She has interacted with Maki (the robot) previously for a different pilot study.
\sect{Task}
For the sake of this exploratory study simple structures as shown in \ref{fig:fig_3-6} are selected as 3D block building tasks. The complex structures(Figure \ref{fig:fig_3-7}) require at least 2 different views to be completed and after initial interaction with the participant, we found that these structures are too complex for her to follow even with feedback from Maki. Thus, we use the simple structures for this study.
\sect{Procedure}
As seen in Figure \ref{fig:fig_1-1}, the setting of tasks consisted of Maki the Robot, playground with blocks, play area and control boxes, a camera mounted on a tripod stand, an iPad for displaying the target 3D block structure and the participant seated on a chair. \\
First of all, the participant is seated. She is presented with a 3D structure to build with blocks as a pre-test for the study. Pre-test is administered to develop an understanding of her existing knowledge of block building. The task given as pre-test is shown in Figure \ref{fig:fig_3-6a}. The experimenter explains the functionality of box controls and play area, and asks the participant to build the given structure . Participant was aware of the functionality partially because of similar interaction in early stages of this project. This prior interaction was carried out to understand if the proposed playground with control boxes is understandable by school-going kids and to analyze the complexity of 3D block building tasks that can be targeted towards elementary-aged students. \\
After pre-test, Maki introduces itself and explains what the participant will be doing today and prompts her to start her first task, as shown in Figure \ref{fig:fig_3-6b}. During the task, Maki provides feedback whenever a mistake is made. On correct actions, Maki narrates statements encouraging to continue or affirming correctness of action e.g., \emph{Okay. Good. Go on. Continue.} etc. so that the participant does not neglect the robot in the process in case she is not making mistakes for long time. The run ends when she successfully completes her task. Maki congratulates her on success and she is asked to take a break before next task with a different target structure. Maki welcomes her back to each next task and prompts her to start the task following same instructions. Four more structures (Figure \ref{fig:fig_3-6c}, \ref{fig:fig_3-6d}, \ref{fig:fig_3-6e} and \ref{fig:fig_3-6f}) are done in next tasks. After all the tasks, a post-test similar to pre-test but with different target structure as shown in Figure \ref{fig:fig_3-6g} is taken. After last task, a post-study interview is conducted in which the experimenter asks the participant about her experience with the system and Maki, whether she found Maki to be helpful or bothersome, whether she would trust Maki to guide her in learning about the given task and if she felt encouraged to continue the task. 

\sect{Results and Discussion}
There are three main sources of data: \textit{pre and post test}, \textit{post- study interview} and \textit{videos} of the interaction. The videos are analyzed by the author. Rudimentary qualitative analysis is done on the video data. We analyzed the pre and post test, videos from 5 tasks and post-study interview qualitatively. Results are discussed here:
\subsect{Pre and Post-test}
The participant successfully completed both pre-test and post-test task. Both tasks were of similar complexity. Overall, outcome for both the tasks was correct but for the pre-test the participant used remove control once self-correcting her mistake. However, in the post-test, she did not make any mistake and only used add control. Also during the pre-test task, experimenter had to explain the functionality of adjust and remove blocks once again. Since there was only one session and tests ended up being rather simple, nothing can be concluded about learning gain from one session qualitatively. \\
\subsect{Video Analysis}
Many observations were made by the viewer of videos of interaction with Maki. The main observations over the 5 tasks are summarized here: 
\begin{enumerate}
    \item The participant looks at Maki when it communicates with her for introduction of task, feedback statements and robot actions (e.g., nodding and referential gaze). But as she goes further into the session, time of her eye contact with Maki reduces while it is repeating the same statements at start and end of each task. But, in the last task, the ending statement was different since Maki tells her about end of her session and bids her goodbye. At the start of this ending statement, she was looking down but as soon as the statement was different from the previous ending statements, she turned her gaze towards Maki to process new information. It is interesting to note that she stopped looking at Maki and started listening passively when it said predictable and repetitive statements but as soon as it narrated new information, she paid full attention to Maki. 
    \item She follows what Maki is asking or suggesting her to do. For instance, during introduction, when Maki says to look at the screen for picture of the target structure, she looks at the screen. When Maki narrates continuers like \emph{Go on, continue}, and so on. she moves on trusting Maki's suggestion that she is correct. She considered \emph{nodding} of Maki as confirmation of her being correct as well.
    \item She looks at robot after every action for confirmation if she did it right or not. Sometimes, she would skip looking at robot but as soon as the robot nods and says a continuer, she would look at robot and move on. Sometimes if robot would not respond to her action, she would wait a couple seconds looking at robot for a response to her action. She would move on if she did not get a confirmation assuming that she did it right. This suggests that if you provide frequent feedback, the user can get accustomed to it and wait for the feedback. Also, it is interesting to note that the lack of feedback made the participant assume she was doing alright. She trusted Maki to point out her mistake. 
    \item During task 4, even though Maki was congratulating her on her successful completion of task, she was looking at the picture of the target structure. She mentioned it to the experimenter that she thought there was still one more block left to be added in the structure.  This might be because of few glitches in the same task. The system made a wrong detection because she took too much time to fix the block and it got detected midway and robot hinted to look at the picture again to see if it is alright. While adjusting again, the robot mentioned that the position is wrong and asked to move the block upwards. This confused the participant since she was placing the block at right place. Experimenter interfered and suggested her to adjust the block again. After adjust action, Maki nodded and suggested to move on since system was able to correct its wrong detection during adjust action. This shows that the participant did not blindly trust Maki and since it made a mistake earlier in the same task, she was skeptical of Maki congratulating her on completion of her task and wanted to make sure for herself.
    \item Apart from task 3, Figure \ref{fig:fig_3-6d}, she did not make mistake in any other task suggesting that the simple tasks (Figure \ref{fig:fig_3-6}) are easy enough for kids of her age group.
    \item Task 3 mistakes: She made 2 mistakes in task 3. The target structure is shown in Figure \ref{fig:fig_3-6d}. The first mistake was that she used a yellow rectangular block instead of a square one at the fourth step. As it is clear from the picture of the target structure, the structure has a possible confusion i.e., \emph{a rectangle of yellow}, instead of \emph{a square of yellow and a square of blue}  at level 2. Maki responded saying \emph{"Are you sure you need a rectangle here?"}. She inferred that the hidden block is blue square not continued yellow rectangular block and she fixed her mistake by removing the yellow rectangular block and replacing it with two square blocks of yellow and blue color. The second mistake was that she placed the blue rectangular block at level 3 before the green square block at the level 2. So, Maki pointed out that the shape should be a square instead of a rectangle. This feedback statement was not very clear i.e., the blue rectangular block at level 3 is the correct block but green square needs to be added at level 2 before that, she corrected her mistake anyways. This implies that merely pointing out that a mistake has occurred nudged the participant to correct herself. Thus, it is safe to say that the participant received feedback from the robot positively and self-corrected her mistakes.
\end{enumerate}
\subsect{Post-study Interview}
To support observations from the video analysis, we conducted a post-study interview. We asked the participant questions about her overall experience, her perception of Maki, her comfortability with the setup, her feelings about Maki's appearance and behaviour. This interview revealed lot of interesting information. Some of the remarks from the participant and our interpretation of these remarks are detailed below:
\begin{enumerate}
    \item \textbf{Fun and engaged}: She thought that the overall experience was fun despite some of the bugs and wrong detections. She also said that she would enjoy more sessions with Maki because it was fun. She expressed that if she had to do the tasks by herself, it would be boring.
    \item \textbf{Fascination}: She was fascinated with Maki since she does not work with robots usually so she thought it was fun. She would prefer working on similar tasks with a robot over a human because she finds that humans can be boring sometimes and \emph{"humans just do not work with other humans"}.
    \item \textbf{Helpfulness}: She perceived Maki to be cool and helpful because she noticed that whenever she was correct, the robot nodded and said \emph{Go on, continue} etc. and whenever she was wrong, it helped her with a hint. She expressed that she would like more sessions with Maki and she learnt a bit about block building task and she will be better at it now. The session boasted her confidence in her ability to work on similar tasks. She expressed that without Maki it would not be easy to complete the tasks.
    \item \textbf{Teacher vs Peer}: She found Maki to be acting like a friend when it gave hints to help her through the task but she thought of Maki as a teacher when it told her if she was wrong or right. This suggests that direct feedback, e.g., using the word \emph{wrong}, creates impression of Maki as a teacher while helpful hints created more of a friend-like personality. 
    \item \textbf{Trust}: She trusted the robot but not blindly. She trusted Maki when it provided feedback on her mistakes. At some point, she thought she was right and the robot was wrong. Even if the robot made mistakes, she was okay with it because Maki was not a human and she could trust it because it was right sometimes. Maki's mistakes in providing feedback would not stop her from working on further sessions with Maki. 
    \item \textbf{Scaffolding and self-regulation}: When Maki used the word \emph{wrong}, her emotions were hurt. She said it made her sad but when Maki acted in a friend-like manner and gave suggestions without using the word wrong, she felt better and she knew that she can correct her mistake. She found Maki to not judge her for her mistakes unlike humans. This is an interesting indication that using softer words to suggest mistakes might have resulted in the participant scaffolding and self-regulating her learning process. 
    \item \textbf{Clarity of feedback}: She found Maki's remarks to be loud and clear. She understood her mistakes from feedback statements and was able to fix those.
    \item \textbf{Frequency of feedback}: The participant was asked if she found Maki to be excessively talkative, annoying and interrupting her. She said that this was not the case. In fact, whenever Maki would not respond she was concerned that it should have said something. 
    \item \textbf{Appearance and robot behaviour}: The loud noise made by robot while moving its head was scary but not in the sense that it would harm her since Maki has no limbs. Whenever Maki moved, she was looking forward to hear something from it. It is interesting to note that robot behaviour such as nod and referential gaze are interpreted as robot's intent to communicate with the participant. The participant was also able to distinguish the meaning of nodding (correct action) and referential gaze (mistake made). Maki's voice was perceived neutral and non-expressive. In her words, it was lacking highs and lows. It did not sound excited or if anything mattered to it. She was of the opinion that Maki should at least be expressive. She did not expect Maki to have emotions as such. 
\end{enumerate}
\sect{Design Implications and Suggested Improvements}
This exploratory study has helped understand the effectiveness and drawbacks of our perception system, feedback strategy and robot behaviour. It has shed light on the potential of robot assisted 3D block building to augment spatial visualization skills in elementary-aged students using implicit feedback strategies to suggest mistakes made in place of explicit corrective feedback. 3D block building task is found to be more engaging and interesting in the presence of the robot. The robot assisted the participant to correct her mistakes through frequent feedback. The participant anticipated feedback from the robot after every action and it concerned her if the robot did not respond. Implicit feedback, such as suggestion, hints and giving her second chance to correct herself, created positive emotions such as confidence in the participant and helped her self-regulate her learning process. Implicit feedback had clarity regardless of not putting the exact mistake in words. Using strong words such as \emph{wrong} had a negative impact on the participant thus, suggesting explicit corrective feedback can harm self-confidence. The robot was able to build some level of trust despite a couple of mistakes on its part. The robot was perceived as a teacher when it communicated whether the action of the participant is right or wrong whereas it was considered a friend when it gave suggestions and hints to assist the participant. Robot's behaviour i.e., nodding and referential gaze were easily distinguishable. New statements or actions by robot would get the participant's full attention whereas repetitive statements caused participant to go into passive listening.\\
This exploratory study will enable us create a framework for a user study to evaluate effectiveness of our improved system and study design. Following are the suggested improvements that will enable us in resolving certain issues of our system and get it ready to launch a full-fledged study to analyze learning gain, self-regulation and scaffolding ability of students and their perception of a robot tutor in one-on-one session while conducting a 3D block building task to augment spatial visualization:
\begin{enumerate}
    \item Two factors need to be part of elaborated study: Type of feedback (implicit vs explicit) and frequency of feedback (when to give feedback). Although given the results from this study, implicit indication of mistake implied better ability to self-regulate and friendliness and likability of the robot, data is too less to make a conclusion. Similarly, the participant preferred frequent feedback in this study but nothing can be said about its effects on learning gain. It is possible that interrupting frequently to provide feedback might make the user too dependent on the robot to figure out when and what mistake is made. Nothing conclusive can be said about these two factors with limited interaction data. These two factors need to be explored further in detail. 
    \item The perception system needs to be improved to increase credibility of feedback provided by the robot.
    \item To be able to evaluate learning gains, ability to scaffold and self-regulate, it would be better to have multiple sessions over a longer period of time since spatial visualization skills a are hard to quantify in one session. Also, these skills improve over time and we are interested in longer-term effects of feedback strategies and robot assistance on spatial visualization. 
    \item Designing the 3D target structures needs to be guided by experts in the field e.g., mathematics teachers, psychologists, and so on. Complexity of pre and post-test needs to be adjusted as well. Standard tests may be conducted to evaluate learning gain in spatial ability. 
    \item Introductory and ending statements provided by the robot at the beginning and end of each task respectively should be made less repetitive to better engage users.  
    \item Loud noises of robot's joints while moving should be minimized to avoid sense of fear in users. Robot voice is neutral and non-expressive in the current system. It could be made more human-like. The robot should sound excited, concerned, happy or disappointed etc. 
\end{enumerate}
Suggested improvements in the proposed system will enable us to progress in this area of research. More studies about the trade\-off between implicit and explicit feedback, effectiveness of various types of implicit feedback and its personalization and right timing for intervention and impact of other various types of feedback strategies etc. will follow.

