\chap{Conclusions and Future Work}
\label{chap:conclusion}
We have demonstrated a system that tracks 3D block building task in real-time. In the \emph{Learning} mode, the system records and stores the assembly process through block additions, removals and adjustments done by the expert (experimenter or teacher etc.) In the \emph{Teaching} mode, a user is prompted to construct a 3D structure provided as a 2D picture. It keeps track of assembly process and detects mistakes made by user and based on those mistakes it provides a selected feedback statement to the user. The system is able to detect shape, color, orientation, level from base and position errors of current block from all possible correct actions. The feedback statements that we have employed mostly fall under the category of implicit and explicit feedback. The feedback is narrated by a robot, Maki. Maki uses nodding with continuers or referential gaze with feedback statements based on type of error that has occurred to inform the user if the user is doing alright or has made a mistake respectively. \\
A brief exploratory study depicts the potential of robot assisted 3D block building task. System was easy to used despite few glitches. The participant perceived Maki to be helpful and essential for task completion. She trusted Maki to guider her despite a couple of poor feedback statements. She was critical of Maki's mistakes indicating absence of blind trust. Maki was perceived as teacher when it used direct feedback whereas it was considered friendly when it presented useful hints and suggestions. Using softer words to implicitly suggest that the participant has made a mistake helped the participant gain confidence in her ability to correct her mistakes by trying again. The feedback statements were loud and clear for the participant to follow easily. The participant became accustomed to frequent feedback. Maki's appearance and behaviour has room for improvement, although, its intentions were easily interpreted by the participant. Predictable statements shifted the participant from paying full attention to listening to Maki passively.\\
There are some limitations of the perception system that can be improved on. We would like to extend this system to work with more shapes, sizes and colors of blocks which is not hard since our system is flexible to include different kinds of blocks. It would be interesting to mover towards algorithms that can handle addition and removal of multiple blocks at a time and allow for merging sub-assemblies. To achieve this, changes might be needed in the model representation. Eventually, it would be interesting to see if similar framework can be extended to more complex building tasks that are not limited to voxelized space.\\
The exploratory study has opened up avenues for multiple studies. One of many possibilities is a study designed to evaluate the impact of implicit vs explicit feedback on self-efficacy and learning gain of school-going students, and their perception of such robot tutor while working on a spatial visualization task. Another multi-session study over a long period of time to evaluate the effect of personalization of feedback strategy on scaffolding and self-regulations for school-going kids for spatial visualization tasks such as 3D block building would prove to be beneficial. It would also be interesting to study the effect of timing, frequency and type of feedback on user's learning gains, self-regulation and perception of robot. Analyzing effect of various non-verbal robot actions such as nodding, referential gaze, hesitation, and so on, as a source of implicitly communicating errors instead of narrated feedback statements could lead to interesting findings. We hope this initial work inspires future research in the field of social robotics and education.  
