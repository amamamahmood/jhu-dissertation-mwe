\chap{Conclusions and Future Work}
\label{chap:conclusion}
We have demonstrated a system that tracks 3D block building task in real-time. In \emph{Learning} mode, the system learns and stores the assembly process through block additions, removals and adjustments done by the expert (experimenter or teacher etc.) In \emph{Teaching} mode, a user is prompted to construct a 3D structure provided as a 2D picture. It keeps track of assembly process and detects mistakes made by user and based on those mistakes it provides a selected feedback statement to the user. The system is able to detect shape, color, orientation, level from base and position errors of current block from all possible correct actions. The feedback statements that we have employed are inspired from literature and mostly fall under the category of implicit and explicit feedback. The feedback is narrated by a robot, Maki. Maki uses nodding with continuers or referential gaze with feedback statements based on type of error that has occurred to imply that the user is doing alright or has made a mistake respectively. \\
A brief exploratory study depicts the potential of robot assisted 3D block building task. System was easy to used despite few glitches. The participant perceived Maki to be helpful and essential for task completion. She trusted Maki to guider her despite a couple of poor feedback statements. She was critical of Maki's mistakes indicating absence of blind trust. Maki was perceived as teacher when he used direct feedback whereas he was considered friendly when he presented useful hints and suggestions. Using softer words to implicitly suggest that the participant has made a mistake helped the participant self-regulate and scaffold her learning process. The feedback statements were loud and clear for the participant to follow easily. The participant became accustomed to frequent feedback. However, Maki's appearance and behaviour has room for improvement, his intentions were easily interpreted by the participant. Predictable statements shifted the participant from paying full attention to Maki to listening to him passively.\\
There are some limitations of the perception system that can be improved on. Future work to improve perception system can target these limitations. We would like to extend this system to work with more shapes, sizes and colors of Lego\textsuperscript\textregistered{} blocks which is not hard since our system is flexible to include more shapes, sizes and colors. It would be interesting to mover towards algorithms that can handle addition and removal of multiple blocks at a time and allow for merging sub-assemblies. To achieve this, changes might be needed in the model representation. Eventually, it would be interesting to see if similar framework can be extended to more complex building tasks that are not limited to voxelized space.\\
The exploratory study has opened up an avenue for multiple studies in this arena. One of many possibilities is a between-subjects study design to evaluate the impact of implicit vs explicit feedback on self-efficacy and learning gain of school-going students, and their perception of such robot tutor while working on a spatial visualization task. Another multi-session study over long period of time to evaluate the effect of personalization of feedback strategy on scaffolding and self-regulations for school-going kids for spatial visualization tasks such as 3D block building would be beneficial addition to research reservoir. It would also be interesting to look at the effect of timing, frequency and type of feedback on learning gains, self-regulation and perception of robot. Analyzing effect of various non-verbal robot actions such as nodding, referential gaze, hesitation etc. as a source of implicitly communicating errors instead of narrated feedback statements could lead to interesting findings. We hope this initial work inspires quality research in the field of social robotics and education.  
